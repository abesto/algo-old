\documentclass{scrartcl}

\makeatletter
\usepackage{tabularx}
\def\hlinewd#1{%
\noalign{\ifnum0=`}\fi\hrule \@height #1 %
\futurelet\reserved@a\@xhline} 
\makeatother

\usepackage[magyar]{babel}
\usepackage[utf8x]{inputenc}
\usepackage{ucs}
\usepackage{amsmath}
\usepackage{amsfonts}
\usepackage{amssymb}
\usepackage{array}
\author{Nagy Zoltán}
\title{Három hasítófüggvény}
\subtitle{EHA kódok ütközésmentes hasítása konstans időben}


\begin{document}
\maketitle
\pagebreak

\section{Feladatkitűzés}
Mérési jegyzőkönyvet készíteni legalább 3 olyan hasító függvényről, amely a csoport EHA kódjait egy elég kicsi méretű olyan hasítótáblára képzi, amiben nincs ütközés.

\section{Megoldás}
Ismertetni fogok három hasítófüggvényt, ami (a bemenő adatok ismeretében) megfelel a feltételeknek. Az EHA kódok közös szuffixumaitól (\textsf{.ELTE}) eltekintek a megoldás során. A hasító\-függvények ismertetése után következik egy összefoglaló táblázat az inputokról és az egyes EHA kódok hasítófüggvény szerinti képeiről. A hasító\-függvényben használt minimális modulusokat és egyéb paramétereket egy "próbálgató" programmal találtam meg.

\subsection{Első függvény}
Kézenfekvő az azonosítókat alkotó karakterek ASCII kódjaival dolgozni. Vehetjük pél\-dá\-ul néhány karakter értékének számtani közepét. Az így kapott függvény érték\-készlete az összes (nagy) betű, de ezt még szűkíthetjük maradékképzéssel. Jelölje $S$ EHA kód $i$-edik karakterének ASCII kódját $S_i$. Legyen az első hasítófüggvény $f_1(S) = \lfloor\frac{S_1 + S_2}{2}\rfloor \;mod \:15$.

\subsection{Második függvény}
Az első függvény kis módosításával kapunk egy nagyobb táblába képző, de még mindig használható függvényt. Optimális index- és modulusválasztásokkal legyen a második hasítófüggvény $f_2(S) = S_1 * S_2 * S_3 \;mod \:23$.

\subsection{Harmadik függvény}
Itt sincs új ötlet, csak a már használt elemeket rendezgetjük. Legyen a harmadik hasítófüggvény $f_3(S) = max(S_1, S_2, S_3) + min(S_1, S_2, S_3) \;mod \:17$.

\subsection{Függvény szerinti képek}
\sffamily
\begin{tabular}{c||c|c|c}
$S$ & $f_1(S)$ & $f_2(S)$ & $f_3(S)$ \\ 
\hlinewd{1pt}
BAMPAAI & 5  & 4  & 12\\
FAANABI & 7  & 16 & 16\\
GEGRAAI & 10 & 0  & 4\\
GUJQAAI & 3  & 22 & 3\\
HAARAAT & 8  & 2  & 1\\
NAZRAAI & 11 & 3  & 7\\
RARRABI & 13 & 14 & 11\\
SCMQAAI & 0  & 6  & 14\\
TACQAAI & 14 & 5  & 13\\
UJZRAAT & 4  & 1  & 6\\
\end{tabular} 
\normalfont

\end{document}